\chapter{Conclusões}
\label{cap:conclusoes}

Vimos, então, que podemos escrever sem muitas dificuldades uma monografia em padrão ABNT sem nos preocuparmos a fundo com a padronização do documento, visto que o \LaTeX nos proporciona a confiabilidade necessária para uma automática padronização.

Indubitavelmente, desta forma, além de pouparmos tempos com regras e rigor podemos ainda manter um documento padrão para a Faculdade de Economia da Universidade Federal Fluminense.

Vimos como inserir tabelas, imagens, sumário e organizar um exemplo de monografia, com esse próprio exemplo de monografia. Por fim, vale ressaltar que o \LaTeX vai \textbf{muito além} deste documento, e suas próprias possibilidades dentro da classe \abnTeX\space ou mesmo da classe \texttt{economia} são infinitas. Assim, é extremamente válido uma busca mais a fundo para pontos de dúvidas. 

A primeira versão desse documento/classe foi concluída no dia 22/04/2019. Pretende-se, entretanto, manter essa classe em desenvolvimento quando necessário bem como em evolução. \textbf{qualquer} dúvida/sugestão, pode ser enviada para \textbf{\texttt{gustavovital@id.uff.br}}. O interesse em colaborar, também é bastante bem vindo e válido.