%------------------------------------------------------------------------------
% Exemplo de documento para auxiliar a padronização das monografias da 
% Faculdade de Economia - UFF
%------------------------------------------------------------------------------

\documentclass[12pt, openright, chapter=TITLE]{economia} %---------------------
% Estão disponíveis opções de impressão como frente e verso, papel A4, outras
% linguagens e ets...
%------------------------------------------------------------------------------ 


%------------------------------------------------------------------------------
% INFORMAÇÕES E DADOS PARA A CAPA
%------------------------------------------------------------------------------

\titulo{Modelo de Monografia em LaTeX  para a Faculdade de Economia UFF Niterói}
\autor{Gustavo de Oliveira Vital}
\data{2019} 
\orientador{Prof. Dr. Jesus Alexei Luizar Obregon}
\coorientador{Prof. Dr$^{a}$. Danielle Carusi Machado}


%------------------------------------------------------------------------------
%	FOLHA DE ROSTO E PREAMBULO
%------------------------------------------------------------------------------

\preambulo{Monografia apresentada ao curso de Bacharelado em Ciências Econômicas da Universidade Federal Fluminense como requisito parcial para conclusão do curso.}

%------------------------------------------------------------------------------
% INÍCIO DO DOCUMENTO
%------------------------------------------------------------------------------

\makeindex

\begin{document}

%------------------------------------------------------------------------------
% ELEMENTOS PRÉ-TEXTUAIS E CAPA
%------------------------------------------------------------------------------

%------------------------------------------------------------------------------
% Aqui iremos inserir, em ordem, a capa '\imprimircapa', a folha de rosto
% '\imprimirfolhaderosto*', as figuras da monografia '\figuras' e as tabelas
% 'tabelas'. É possível também a inserção de quadros, por exemplo, com o 
% comando 'quadros'. Sempre pulando uma linha entre as opções de ''listas''
%------------------------------------------------------------------------------

\imprimircapa 						
\imprimirfolhaderosto*	  					

\include{aprovacao}	
\clearpage

\input{capitulos/resumo.tex}	

\input{capitulos/dedicatoria}		
					
\figuras

\tabelas

\sumario


% -----------------------------------------------------------------------------
% ELEMENTOS TEXTUAIS
% -----------------------------------------------------------------------------

\textual



% ------------------------------------------------------------------
% Exemplo de introdução gerada por textos dummys a partir do
% lipsum
%-------------------------------------------------------------------


\chapter{Introdução}
\label{cap:intro} % faço a referência na bibliografia

\lipsum[1]

%----------------------------------------------------------------------
\section{Motivação}
\label{sec:motivacao}

\lipsum[2-4]

%----------------------------------------------------------------------
\subsection{Objetivos}
\label{sec:objetivos}

\lipsum[2-5]

%----------------------------------------------------------------------
\section{Estrutura do Trabalho}
\label{sec:estrutura}

\lipsum[1]

\input{capitulos/graficos.tex}

\input{capitulos/tabelas.tex}

\input{capitulos/comocitar.tex}

\input{capitulos/conclusao.tex}




% ----------------------------------------------------------
% ELEMENTOS PÓS-TEXTUAIS
% ----------------------------------------------------------
\postextual

% ----------------------------------------------------------
% Referências bibliográficas
% ----------------------------------------------------------

\nocite{hobsbawm1995era,
		battista2000reduction,
		abntexmanual, 
		abntex2modelo} % aqui citamos os autores que não foram citados no texto.

\bibliography{referencias}


\end{document}
